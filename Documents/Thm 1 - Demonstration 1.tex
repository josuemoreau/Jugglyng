\documentclass[a4paper, 11pt]{article}

\usepackage[utf8]{inputenc}
\usepackage[T1]{fontenc}
\usepackage[french]{babel}
\usepackage[margin=2.5cm]{geometry}

\usepackage{amsmath}
\usepackage{amsfonts}
\usepackage{amssymb}
\usepackage{amsthm}
\usepackage{graphics}
\usepackage{enumitem}
\usepackage{frame}

% \title{Rapport de stage\\\normalsize{Automates et jonglerie musicale}}
\title{Automates et jonglerie musicale}
\author{Josué Moreau}

\theoremstyle{plain}
\newtheorem{thm}{Théorème}[]

\begin{document}

\maketitle

\begin{thm}
  Soit $\mathcal{A}$ un automate de jonglage à $b$ balles et de hauteur maximale
  $h$. Soit une séquence $s$ de longueur au moins $h - 1$. Alors, soit $s$
  apparaît une unique fois dans $\mathcal{A}$, soit $s$ n'apparaît pas dans
  $\mathcal{A}$.
\end{thm}

\begin{proof}
  Soit $\mathcal{A}$ un automate de jonglage à $b$ balles et de hauteur maximale
  $h$. Soit une séquence $s$ de longueur au moins $h - 1$.
  \begin{itemize}
  \item Si $s$ n'apparaît pas dans $\mathcal{A}$, alors on a bien le résultat
    voulu.
  \item Si $s$ apparaît dans $\mathcal{A}$. Nécessairement $s$ se joue avec
    exactement $b$ balles.
    \begin{enumerate}
    \item Si $s$ est de longueur au moins $h$. Posons, pour tout
      $0 \leq i < |s| + h$ :
      \[
        s_i^t =
        \left\{
          \begin{array}{ll}
            0 & \text{s'il existe } t \leq j < i \text{ tel que } j + s_j = i\\
            s_i & \text{sinon}
          \end{array}
        \right.
      \]
      et
      \[
        b_i^t =
        \left\{
          \begin{array}{ll}
            0 & \text{si } s_i^t = 0\\
            1 & \text{sinon}
          \end{array}
        \right.
      \]
      On peut alors définir $e_t = b_t^t\;b_{t + 1}^t\; ... \;b_{t + h - 1}^t$.
      On a bien $e_t \xrightarrow{s_t} e_{t + 1}$ dans $\mathcal{A}$ car :
      \begin{itemize}
      \item Si $s_t = 0$, alors $s_{t + s_t}^{t} = s_t^t = s_t = 0$. De plus,
        $s_{t + h}^t = 0$ car une balle atterrit nécessairement au temps
        $t + h$, sinon on introduirait une nouvelle balle, ce qui engendrerait
        une séquence à plus de $b$ balles, non représentable dans l'automate
        $\mathcal{A}$. Donc $e_{t + 1}$ est égal à $e_t$ auquel on a ajouté un
        $0$ à droite et enlevé $b_t^t$. Ainsi, la transition
        $e_t \xrightarrow{s_t} e_{t + 1}$ est bien dans $\mathcal{A}$.
      \item Si $s_t > 0$.
        \begin{itemize}
        \item $s_{t + s_t}^t = 0$ donc $b_{t + s_t}^t = 0$ donc, si $s_t < h$,
          $e_t[s_t] = 0$.
        \item $s_{t + s_t}^{t + 1} = s_{t + s_t} \neq 0$ car sinon
          $s_{t + s_t} = 0$ et alors $s$ ne serait pas une séquence de jonglage
          (la balle lancée en $t$ sur une hauteur $s_t$ ne serait pas relancée
          en $t + s_t$). Donc, $b_{t + s_t}^{t + 1} = 1$ donc
          $e_{t + 1}[s_t - 1] = 1$. Ainsi, on a bien
          $e_t \xrightarrow{s_t} e_{t + 1}$.
        \end{itemize}
      \end{itemize}
      Il n'existe pas d'autre suite d'état dans l'automate $\mathcal{A}$ qui
      représente la même séquence. En effet, dans le cas contraire, il
      existerait un état $e_t'$ pour un certain $t$ tel que $e_t' \neq e_t$.
      Donc, il existerait $0 \leq i < h$ tel que $e_t'[i] = 1$ et $e_t[i] = 0$.
      Donc il existerait $t \leq j < t + i$ tel que $j + s_j = t + i$ et
      $e_t'[i] = 1$. On aurait alors un conflit car une balle va atterrir au
      temps $t + i$ alors qu'une autre balle va être lancée au temps $t + j$
      pour atterrir aussi au temps $t + i$. Ainsi, $e_0, ..., e_|s|$ est unique.
    \item Si $s$ est de longueur $h - 1$. Alors,
      \begin{itemize}
      \item Si elle se joue à $b$ balles. Alors, il suffit, pour tout
        $0 \leq i < h$, de mettre un $0$ en position $h - i - 1$ de l'état
        $e_i$. La même séquence avec des nombres différents de $0$ à ces
        emplacement n'est pas dans $\mathcal{A}$ car elle nécessiterait $b + 1$
        balles ou comporterait des conflits.
      \item Si elle se joue à $b - 1$ balles, alors on fait la même chose qu'au
        point précédent avec uniquement des $1$. La même séquence avec des
        nombres différents de $1$ à ces emplacements n'est pas dans
        $\mathcal{A}$ car elle nécessiterait alors $b - 1$ balles ou
        comporterait des conflits.
      \end{itemize}
    \end{enumerate}
  \end{itemize}
  Ainsi, soit une séquence $s$ de longueur au moins $h - 1$, $s$ apparaît une
  unique fois dans $\mathcal{A}$ ou n'apparaît pas dans $\mathcal{A}$.
\end{proof}

\end{document}

% LocalWords:  jonglage
